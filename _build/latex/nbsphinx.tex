%% Generated by Sphinx.
\def\sphinxdocclass{report}
\documentclass[letterpaper,10pt,english]{sphinxmanual}
\ifdefined\pdfpxdimen
   \let\sphinxpxdimen\pdfpxdimen\else\newdimen\sphinxpxdimen
\fi \sphinxpxdimen=.75bp\relax

\PassOptionsToPackage{warn}{textcomp}
\usepackage[utf8]{inputenc}
\ifdefined\DeclareUnicodeCharacter
% support both utf8 and utf8x syntaxes
  \ifdefined\DeclareUnicodeCharacterAsOptional
    \def\sphinxDUC#1{\DeclareUnicodeCharacter{"#1}}
  \else
    \let\sphinxDUC\DeclareUnicodeCharacter
  \fi
  \sphinxDUC{00A0}{\nobreakspace}
  \sphinxDUC{2500}{\sphinxunichar{2500}}
  \sphinxDUC{2502}{\sphinxunichar{2502}}
  \sphinxDUC{2514}{\sphinxunichar{2514}}
  \sphinxDUC{251C}{\sphinxunichar{251C}}
  \sphinxDUC{2572}{\textbackslash}
\fi
\usepackage{cmap}
\usepackage[T1]{fontenc}
\usepackage{amsmath,amssymb,amstext}
\usepackage{babel}



\usepackage{times}
\expandafter\ifx\csname T@LGR\endcsname\relax
\else
% LGR was declared as font encoding
  \substitutefont{LGR}{\rmdefault}{cmr}
  \substitutefont{LGR}{\sfdefault}{cmss}
  \substitutefont{LGR}{\ttdefault}{cmtt}
\fi
\expandafter\ifx\csname T@X2\endcsname\relax
  \expandafter\ifx\csname T@T2A\endcsname\relax
  \else
  % T2A was declared as font encoding
    \substitutefont{T2A}{\rmdefault}{cmr}
    \substitutefont{T2A}{\sfdefault}{cmss}
    \substitutefont{T2A}{\ttdefault}{cmtt}
  \fi
\else
% X2 was declared as font encoding
  \substitutefont{X2}{\rmdefault}{cmr}
  \substitutefont{X2}{\sfdefault}{cmss}
  \substitutefont{X2}{\ttdefault}{cmtt}
\fi


\usepackage[Bjarne]{fncychap}
\usepackage{sphinx}

\fvset{fontsize=\small}
\usepackage{geometry}


% Include hyperref last.
\usepackage{hyperref}
% Fix anchor placement for figures with captions.
\usepackage{hypcap}% it must be loaded after hyperref.
% Set up styles of URL: it should be placed after hyperref.
\urlstyle{same}

\usepackage{sphinxmessages}
\setcounter{tocdepth}{1}


% Jupyter Notebook code cell colors
\definecolor{nbsphinxin}{HTML}{307FC1}
\definecolor{nbsphinxout}{HTML}{BF5B3D}
\definecolor{nbsphinx-code-bg}{HTML}{F5F5F5}
\definecolor{nbsphinx-code-border}{HTML}{E0E0E0}
\definecolor{nbsphinx-stderr}{HTML}{FFDDDD}
% ANSI colors for output streams and traceback highlighting
\definecolor{ansi-black}{HTML}{3E424D}
\definecolor{ansi-black-intense}{HTML}{282C36}
\definecolor{ansi-red}{HTML}{E75C58}
\definecolor{ansi-red-intense}{HTML}{B22B31}
\definecolor{ansi-green}{HTML}{00A250}
\definecolor{ansi-green-intense}{HTML}{007427}
\definecolor{ansi-yellow}{HTML}{DDB62B}
\definecolor{ansi-yellow-intense}{HTML}{B27D12}
\definecolor{ansi-blue}{HTML}{208FFB}
\definecolor{ansi-blue-intense}{HTML}{0065CA}
\definecolor{ansi-magenta}{HTML}{D160C4}
\definecolor{ansi-magenta-intense}{HTML}{A03196}
\definecolor{ansi-cyan}{HTML}{60C6C8}
\definecolor{ansi-cyan-intense}{HTML}{258F8F}
\definecolor{ansi-white}{HTML}{C5C1B4}
\definecolor{ansi-white-intense}{HTML}{A1A6B2}
\definecolor{ansi-default-inverse-fg}{HTML}{FFFFFF}
\definecolor{ansi-default-inverse-bg}{HTML}{000000}

% Define an environment for non-plain-text code cell outputs (e.g. images)
\makeatletter
\newenvironment{nbsphinxfancyoutput}{%
    % Avoid fatal error with framed.sty if graphics too long to fit on one page
    \let\sphinxincludegraphics\nbsphinxincludegraphics
    \nbsphinx@image@maxheight\textheight
    \advance\nbsphinx@image@maxheight -2\fboxsep   % default \fboxsep 3pt
    \advance\nbsphinx@image@maxheight -2\fboxrule  % default \fboxrule 0.4pt
    \advance\nbsphinx@image@maxheight -\baselineskip
\def\nbsphinxfcolorbox{\spx@fcolorbox{nbsphinx-code-border}{white}}%
\def\FrameCommand{\nbsphinxfcolorbox\nbsphinxfancyaddprompt\@empty}%
\def\FirstFrameCommand{\nbsphinxfcolorbox\nbsphinxfancyaddprompt\sphinxVerbatim@Continues}%
\def\MidFrameCommand{\nbsphinxfcolorbox\sphinxVerbatim@Continued\sphinxVerbatim@Continues}%
\def\LastFrameCommand{\nbsphinxfcolorbox\sphinxVerbatim@Continued\@empty}%
\MakeFramed{\advance\hsize-\width\@totalleftmargin\z@\linewidth\hsize\@setminipage}%
\lineskip=1ex\lineskiplimit=1ex\raggedright%
}{\par\unskip\@minipagefalse\endMakeFramed}
\makeatother
\newbox\nbsphinxpromptbox
\def\nbsphinxfancyaddprompt{\ifvoid\nbsphinxpromptbox\else
    \kern\fboxrule\kern\fboxsep
    \copy\nbsphinxpromptbox
    \kern-\ht\nbsphinxpromptbox\kern-\dp\nbsphinxpromptbox
    \kern-\fboxsep\kern-\fboxrule\nointerlineskip
    \fi}
\newlength\nbsphinxcodecellspacing
\setlength{\nbsphinxcodecellspacing}{0pt}

% Define support macros for attaching opening and closing lines to notebooks
\newsavebox\nbsphinxbox
\makeatletter
\newcommand{\nbsphinxstartnotebook}[1]{%
    \par
    % measure needed space
    \setbox\nbsphinxbox\vtop{{#1\par}}
    % reserve some space at bottom of page, else start new page
    \needspace{\dimexpr2.5\baselineskip+\ht\nbsphinxbox+\dp\nbsphinxbox}
    % mimick vertical spacing from \section command
      \addpenalty\@secpenalty
      \@tempskipa 3.5ex \@plus 1ex \@minus .2ex\relax
      \addvspace\@tempskipa
      {\Large\@tempskipa\baselineskip
             \advance\@tempskipa-\prevdepth
             \advance\@tempskipa-\ht\nbsphinxbox
             \ifdim\@tempskipa>\z@
               \vskip \@tempskipa
             \fi}
    \unvbox\nbsphinxbox
    % if notebook starts with a \section, prevent it from adding extra space
    \@nobreaktrue\everypar{\@nobreakfalse\everypar{}}%
    % compensate the parskip which will get inserted by next paragraph
    \nobreak\vskip-\parskip
    % do not break here
    \nobreak
}% end of \nbsphinxstartnotebook

\newcommand{\nbsphinxstopnotebook}[1]{%
    \par
    % measure needed space
    \setbox\nbsphinxbox\vbox{{#1\par}}
    \nobreak % it updates page totals
    \dimen@\pagegoal
    \advance\dimen@-\pagetotal \advance\dimen@-\pagedepth
    \advance\dimen@-\ht\nbsphinxbox \advance\dimen@-\dp\nbsphinxbox
    \ifdim\dimen@<\z@
      % little space left
      \unvbox\nbsphinxbox
      \kern-.8\baselineskip
      \nobreak\vskip\z@\@plus1fil
      \penalty100
      \vskip\z@\@plus-1fil
      \kern.8\baselineskip
    \else
      \unvbox\nbsphinxbox
    \fi
}% end of \nbsphinxstopnotebook

% Ensure height of an included graphics fits in nbsphinxfancyoutput frame
\newdimen\nbsphinx@image@maxheight % set in nbsphinxfancyoutput environment
\newcommand*{\nbsphinxincludegraphics}[2][]{%
    \gdef\spx@includegraphics@options{#1}%
    \setbox\spx@image@box\hbox{\includegraphics[#1,draft]{#2}}%
    \in@false
    \ifdim \wd\spx@image@box>\linewidth
      \g@addto@macro\spx@includegraphics@options{,width=\linewidth}%
      \in@true
    \fi
    % no rotation, no need to worry about depth
    \ifdim \ht\spx@image@box>\nbsphinx@image@maxheight
      \g@addto@macro\spx@includegraphics@options{,height=\nbsphinx@image@maxheight}%
      \in@true
    \fi
    \ifin@
      \g@addto@macro\spx@includegraphics@options{,keepaspectratio}%
    \fi
    \setbox\spx@image@box\box\voidb@x % clear memory
    \expandafter\includegraphics\expandafter[\spx@includegraphics@options]{#2}%
}% end of "\MakeFrame"-safe variant of \sphinxincludegraphics
\makeatother

\makeatletter
\renewcommand*\sphinx@verbatim@nolig@list{\do\'\do\`}
\begingroup
\catcode`'=\active
\let\nbsphinx@noligs\@noligs
\g@addto@macro\nbsphinx@noligs{\let'\PYGZsq}
\endgroup
\makeatother
\renewcommand*\sphinxbreaksbeforeactivelist{\do\<\do\"\do\'}
\renewcommand*\sphinxbreaksafteractivelist{\do\.\do\,\do\:\do\;\do\?\do\!\do\/\do\>\do\-}
\makeatletter
\fvset{codes*=\sphinxbreaksattexescapedchars\do\^\^\let\@noligs\nbsphinx@noligs}
\makeatother



\title{Assessment in secondary geography}
\date{Mar 11, 2020}
\release{}
\author{}
\newcommand{\sphinxlogo}{\vbox{}}
\renewcommand{\releasename}{}
\makeindex
\begin{document}

\pagestyle{empty}
\sphinxmaketitle
\pagestyle{plain}
\sphinxtableofcontents
\pagestyle{normal}
\phantomsection\label{\detokenize{index::doc}}


Content generated from the OpenLearn Unit \sphinxhref{https://www.open.edu/openlearn/education/assessment-secondary-geography/content-section-0}{Assessment in secondary geography}.


\chapter{Contents:}
\label{\detokenize{index:contents}}

\section{Session 00}
\label{\detokenize{index:session-00}}

\subsection{1 What does it mean to make progress in geography?}
\label{\detokenize{content/session_00/Part_00_01:1-What-does-it-mean-to-make-progress-in-geography?}}\label{\detokenize{content/session_00/Part_00_01::doc}}
As a teacher, it is your responsibility to help students to make progress in learning but what exactly constitutes progress?


\subsubsection{Activity 1 Defining progress}
\label{\detokenize{content/session_00/Part_00_01:Activity-1-Defining-progress}}
\sphinxstylestrong{Timing: Allow about 10 minutes}


\paragraph{Question}
\label{\detokenize{content/session_00/Part_00_01:Question}}\begin{itemize}
\item {} 
Thinking about your own geographical learning and that of students, how would you define progress?

\item {} 
Can you identify and explain any difficulties in ‘pinning down’ exactly what progression means?

\item {} 
Are you aware of other educational stakeholders who would define progress in different ways? If so: Who are they?What are their criteria? Why do they differ?

\end{itemize}

Jot down your thoughts before reading on.

Taylor (2013, p. 302) notes:


\begin{quote}

Learning involves change in someone’s knowledge, understanding, skills or attitudes (and the meaning of each of those terms and the relationship between them is complex and contested), but a neutral idea of ‘change’ is not good enough. The change must be seen as valuable, as moving in a positive direction, as progress.
\end{quote}

This raises the question: who decides what is valuable? The learner, the teacher, parents, schools, examination boards, government departments or regulators? Different stakeholders have different ideologies. In a subject as broad as geography, what is valued by one person might not be by another. The emphasis placed on knowledge, understanding, skills and attitudes varies. Geography is also dynamic and continually under construction, so the valued elements also change over time.


\subparagraph{1.1 Notions of progression}
\label{\detokenize{content/session_00/Part_00_01:1.1-Notions-of-progression}}
Progression takes place over a number of timescales: across a sequence of lessons; a year; a key stage. You have to decide how to structure teaching and assessment over these timescales in order to support students’ progression in learning.

Progression can be considered in relation to the learning experiences planned by the teacher:
\begin{itemize}
\item {} 
increasing breadth of study

\item {} 
wider range of scales studied

\item {} 
greater complexity of phenomena studied

\item {} 
introducing more precise geographical terminology and vocabulary

\item {} 
increasing use of generalised knowledge and abstract ideas, particularly through making connections and comprehending relationships (Weeden, 2013, p. 147)

\item {} 
requiring greater precision in undertaking intellectual and practical tasks

\item {} 
a more mature awareness and understanding of issues and of the context of differing attitudes and values within which they arise.

\end{itemize}

Progress will be apparent if students are ready for more demanding teaching and learning experiences. Progression can also be viewed in terms of student performance (achievement against mark schemes and grade criteria for specific units of work or GCSE or A Levels).

To help students make progress, you need to have a clear understanding of the learning that they need to do, where they are now and how best to help them bridge the gap. Assessment is, therefore, a key component of planning for progression. Individual students vary in the rate at which they progress, according to their individual interests and abilities, the learning and teaching styles employed and the nature and structure of the activity. Assessment should be designed so that all students have
an opportunity to demonstrate their progress and so that you can plan to address individuals’ learning needs.

The following activity allows you to consider proposals from the Geographical Association (GA) for integrated planning for progression and assessment.


\subsubsection{Activity 2 Assessment and progression \textendash{} views from the GA}
\label{\detokenize{content/session_00/Part_00_01:Activity-2-Assessment-and-progression-_-views-from-the-GA}}
\sphinxstylestrong{Timing: Allow about 1 hour}


\paragraph{Question}
\label{\detokenize{content/session_00/Part_00_01:id1}}
In England, the Geographical Association is concerned that an emphasis on locational and place knowledge, human and physical processes in the Geography National Curriculum (GNC) 2014 will erode notions of progression in curriculum and assessment planning. They have published a framework to address their concerns. The discussion of the proposals related to assessment and progression will be relevant to teachers whether they work in England or elsewhere.
\begin{enumerate}
\sphinxsetlistlabels{\arabic}{enumi}{enumii}{}{.}%
\item {} 
If you want to know more about the context of assessment and progression in England prior to the introduction of the GNC in 2014, read Lambert’s (2010) think piece on progression, which has a short discussion and some useful appendices.

\item {} 
Download and save An assessment and progression framework for geography (Geographical Association, 2014a) and the accompanying PowerPoint presentation, Progression and assessment without levels (Geographical Association, 2014b). You will use these again in Activity 4.

\end{enumerate}

Read the following sections: ‘A clear vision’ and ‘The framework’. Also look at the framework diagram on page 4. Read slides 1\textendash{}12 and 28 in the PowerPoint presentation and the accompanying notes.

What does the Geographical Association see as possible benefits of the framework advocated here?


\paragraph{Discussion}
\label{\detokenize{content/session_00/Part_00_01:Discussion}}
The Geographical Association says:
\begin{itemize}
\item {} 
The framework seeks to address concerns that an emphasis on graded assessments encouraged a labelling culture (students saw themselves as being a Level 4b, etc.). This could place a ‘ceiling’ on student progress.

\item {} 
The use of levels can encourage teachers and students to ‘race through’ the levels and see learning as a linear process.

\end{itemize}

Emphasis is placed on the following:
\begin{itemize}
\item {} 
progress through ‘revisiting places and topics in ways that \sphinxstylestrong{build depth of knowledge and understanding} rather than a simple step\sphinxhyphen{}by\sphinxhyphen{}step process’

\item {} 
\sphinxstylestrong{decisions at the school level} in terms of assessment policies and teachers’ roles in contextualising expectations to develop assessment criteria that will make sense to students.

\end{itemize}


\subparagraph{1.2 Summary}
\label{\detokenize{content/session_00/Part_00_01:1.2-Summary}}
There is no definitive definition of what progress in geography means. Often, progress is assessed in terms of student performance related to level descriptors and examination marking criteria. To support learning you might adopt a broader consideration of progress that focuses on deepening and securing knowledge and understanding, and making connections rather than encouraging students to ‘tick one box’ and move on to the next challenge.


\subsection{2 How can assessment be integrated into teaching and learning?}
\label{\detokenize{content/session_00/Part_00_02:2-How-can-assessment-be-integrated-into-teaching-and-learning?}}\label{\detokenize{content/session_00/Part_00_02::doc}}
This section considers how a range of purposes determines the focus and implementation of assessment and how you might plan to integrate assessment into learning sequences. The following activity will help you to consider your own experiences of, and ideas about, assessment.


\subsubsection{Activity 3 Your personal views of assessment}
\label{\detokenize{content/session_00/Part_00_02:Activity-3-Your-personal-views-of-assessment}}
\sphinxstylestrong{Timing: Allow about 15 minutes}


\paragraph{Question}
\label{\detokenize{content/session_00/Part_00_02:Question}}
Note down what comes to mind when you think about assessment. The following questions will help to prompt your thinking:
\begin{itemize}
\item {} 
What constitutes assessment?

\item {} 
What experiences of assessment have you had?

\item {} 
What were the purposes of different types of assessment?

\item {} 
What feelings do you associate with assessment?

\item {} 
Which experiences of assessment have been most helpful to you as a learner and in what ways?

\end{itemize}


\paragraph{Question}
\label{\detokenize{content/session_00/Part_00_02:question-1}}\label{\detokenize{content/session_00/Part_00_02:id1}}
Make lists of what you think assessment should achieve from the perspective of the:
\begin{itemize}
\item {} 
learner

\item {} 
parent/carer

\item {} 
teacher

\item {} 
school

\item {} 
government, employers and wider society.

\end{itemize}


\paragraph{Question}
\label{\detokenize{content/session_00/Part_00_02:question-2}}\label{\detokenize{content/session_00/Part_00_02:id2}}
Which do you think are the most important purposes of assessment and why?


\paragraph{Discussion}
\label{\detokenize{content/session_00/Part_00_02:Discussion}}
You may have thought of timed, written tests that were intended to sum up whatever you had learned over a course, the results of which determined setting and further study paths. Associated with these memories there may be feelings of anxiety or elation. As a successful learner you will have learned to cope with such assessments. You may have found them motivational. For those who find learning more difficult, or who had an off\sphinxhyphen{}day, however, assessment may impact on their self\sphinxhyphen{}confidence and
self\sphinxhyphen{}worth, as well as their attitude towards geography and towards school.

The purposes of assessment are discussed on the next page.


\subparagraph{2.1 Roles of assessment}
\label{\detokenize{content/session_00/Part_00_02:2.1-Roles-of-assessment}}
More often than not, assessment in the past was done \sphinxstylestrong{to} students. Its main purpose was not to help improve learning; it was to find out what someone did \textendash{} or did not \textendash{} know. Learners were often simply given a mark or a brief comment (‘excellent’, ‘could do better’ and so on). Little help was given to learners regarding their particular difficulties or how they might improve their work. This is assessment of learning (AoL), which is also known as summative assessment.

The role of assessment in school today is thought about very differently \textendash{} much emphasis is placed on assessment for learning (AfL, also known as formative assessment). Links between different types of assessment are summarised in Figure 1, and Table 1 distinguishes between AoL and AfL.

\sphinxincludegraphics[width=498\sphinxpxdimen,height=205\sphinxpxdimen]{{pgce_geography_assessment_fig01}.gif}

Figure 1 Links between different types of assessment





Table 1 A summary of AoL and AfL













Assessment \sphinxstyleemphasis{of} learning





Assessment \sphinxstyleemphasis{for} learning









\sphinxstylestrong{Purpose}





To find out what students know, understand and can do (skills)





To find out what students know, understand and can do (skills)









\sphinxstylestrong{Uses}





Medium and long\sphinxhyphen{}term reporting to others (exam results)

Judging school effectiveness

Certification





Day to day and within sequences of lessons, assessment \sphinxstylestrong{for} {\color{red}\bfseries{}and\_}\_ with \_\_students

To inform planning (future learning objectives)

To support students in making and monitoring progress (identify individual, evolving needs)

To support teacher evaluation of the effectiveness of practice (check objectives against outcomes)









\sphinxstylestrong{Audiences}





Parents

School

External agencies (e.g. inspections)





Students

Teacher









An over\sphinxhyphen{}reliance on AoL \textendash{} focusing on grades rather than where to go next \textendash{} can damage the self\sphinxhyphen{}esteem of low attainers, while high attainers become reluctant to take risks due to fear of failure. AfL approaches seek to avoid these pitfalls. The importance of AfL and the crucial role that it plays in enhancing teaching and learning is supported by a great deal of research. It is important to ensure that assessment is the servant of teaching and learning and not their master.


\subsubsection{Reflection point}
\label{\detokenize{content/session_00/Part_00_02:Reflection-point}}
Can summative assessment be used formatively?

If assessment is to support learning, then how it is carried out is of prime importance. It is not simply what a teacher does but also how it is done that is important. To make AfL integral to learning try to:
\begin{itemize}
\item {} 
\sphinxstylestrong{Identify what you want students to learn; use appropriate tasks and blend in activities that will allow AfL} in terms of individuals’ techniques and skills, understanding of wider concepts and ability to apply their understanding in less familiar contexts.

\item {} 
\sphinxstylestrong{Assess students as individuals} as they learn at different rates; they learn different things from each other and from their experiences; and some develop misconceptions. Then teaching can be matched to needs of individuals. Include opportunities for students to respond to feedback.

\item {} 
\sphinxstylestrong{Integrate continuous, systematic assessment into every lesson} so that evidence is used to adapt teaching to meet learning needs \textendash{} in this way it becomes AfL (Black et al., 2004, p. 10).

\item {} 
\sphinxstylestrong{Assess the effectiveness of your teaching}: use evidence about the effectiveness of the teaching decisions that you take (approaches, resources, tasks, activities, timings and so on) to inform modifications within lessons or sequences.

\end{itemize}

Without the process of adapting teaching and learning, the assessment is not formative \textendash{} it is merely frequent.


\subsubsection{Activity 4 Integrating AoL and AfL}
\label{\detokenize{content/session_00/Part_00_02:Activity-4-Integrating-AoL-and-AfL}}
\sphinxstylestrong{Timing: Allow about 1 hour 15 minutes}


\paragraph{Question}
\label{\detokenize{content/session_00/Part_00_02:id3}}
Revisit the files from the GA web page that you saved for Section 1 Activity 2.
\begin{itemize}
\item {} 
An assessment and progression framework for geography: read the last two sections.

\item {} 
Progression and assessment without levels (PowerPoint presentation): read slides 17\textendash{}27 and the accompanying notes.

\end{itemize}


\paragraph{Question}
\label{\detokenize{content/session_00/Part_00_02:id4}}\label{\detokenize{content/session_00/Part_00_02:id5}}
The PowerPoint presentation includes some examples based on a tectonics unit. Look at slide 13 to pick out ‘what to assess’, using the age\sphinxhyphen{}related benchmark expectations from page 4 of the \sphinxstyleemphasis{Framework}pdf.

Think about how this will translate into ‘how’, using slides 23\textendash{}26.


\paragraph{Question}
\label{\detokenize{content/session_00/Part_00_02:id6}}\label{\detokenize{content/session_00/Part_00_02:id7}}
Thinking about the shorter term, read Sidhu (2011), Why use AfL? Dusting off the black box.
\begin{itemize}
\item {} 
How do his experiences relate to the four tips for making AfL integral to teaching ?

\end{itemize}


\paragraph{Question}
\label{\detokenize{content/session_00/Part_00_02:id8}}
\sphinxstylestrong{Identifying what you want students to learn}


\paragraph{Question}
\label{\detokenize{content/session_00/Part_00_02:id9}}
\sphinxstylestrong{Assessing students as individuals}


\paragraph{Question}
\label{\detokenize{content/session_00/Part_00_02:id10}}\label{\detokenize{content/session_00/Part_00_02:id11}}
\sphinxstylestrong{Integrating assessment into every lesson}


\paragraph{Question}
\label{\detokenize{content/session_00/Part_00_02:id12}}
\sphinxstylestrong{Assessing the effectiveness of your teaching:}


\subparagraph{2.3 Summary}
\label{\detokenize{content/session_00/Part_00_02:2.3-Summary}}
While AoL has important roles to play in the education system, AfL can foster a sense of interdependence in the classroom and help to empower and engage students (Sidhu, 2011). In terms of how to assess, you need to have a long\sphinxhyphen{}term strategy to integrate AfL and AoL related to progression within the geography curriculum. Medium\sphinxhyphen{} and short\sphinxhyphen{}term assessment can then be devised to give all students the opportunity to demonstrate progress and give you the opportunity to respond to learners’ needs.


\subsection{3 How can assessment support learning?}
\label{\detokenize{content/session_00/Part_00_03:3-How-can-assessment-support-learning?}}\label{\detokenize{content/session_00/Part_00_03::doc}}
In this section you will consider the roles of questioning and feedback, and how involving students in the assessment process supports them in becoming independent learners.


\subsubsection{3.1 Questioning}
\label{\detokenize{content/session_00/Part_00_03:3.1-Questioning}}
You may ask students questions to develop a narrative of the lesson or to reinforce instructions and learning intentions as a key management tool. Other questions are intended to assess knowledge, understanding and skills. These assessment questions may range in their cognitive challenge and may be either closed or open (Figure 2).

\sphinxincludegraphics[width=512\sphinxpxdimen,height=410\sphinxpxdimen]{{pgce_geography_assessment_fig02}.gif}

Figure 2 Two dimensions of questioning (Lambert and Balderstone, 2010, p. 105)

Aim to make your questioning inclusive. Planning questions before teaching a lesson and tailoring these questions to students within your classes can build student motivation. Remember there are different types of question that elicit different types of response. Try to stretch and challenge all of your students. Try not to put a ceiling on what they might be able to achieve with a little support (or scaffolding).


\paragraph{Activity 5 Dimensions of questioning}
\label{\detokenize{content/session_00/Part_00_03:Activity-5-Dimensions-of-questioning}}
\sphinxstylestrong{Timing: Allow about 45 minutes}


\subparagraph{Question}
\label{\detokenize{content/session_00/Part_00_03:Question}}
Print Figure 2 and annotate the diagram to show your ideas on:
\begin{itemize}
\item {} 
which students would benefit from the different types of question

\item {} 
when it might be appropriate to ask the different types of question.

\end{itemize}


\subparagraph{Question}
\label{\detokenize{content/session_00/Part_00_03:question-1}}\label{\detokenize{content/session_00/Part_00_03:id1}}
For a lesson you are planning, or using the Year 12 lesson plan, devise a series of questions that you could pose, which vary in terms of cognitive challenge and closed/open nature. Try to list a minimum of twelve questions and then categorise these questions using Bloom’s revised taxonomy (Figure 3).

\sphinxincludegraphics[width=504\sphinxpxdimen,height=396\sphinxpxdimen]{{pgce_geography_assessment_fig03}.gif}

Figure 3 Bloom’s revised taxonomy

Practical strategies for effective questioning for AfL include:
\begin{itemize}
\item {} 
‘no hands up’ to help to involve all students

\item {} 
‘hot\sphinxhyphen{}seating’ to extend interaction with one student and to scaffold learning and drill down into knowledge and understanding (this can also include students posing questions)

\item {} 
allowing sufficient wait time for a student to respond and, crucially for you to feedback, take some time to evaluate student responses, especially if the answer was unexpected

\item {} 
using ‘pose\sphinxhyphen{}pause\sphinxhyphen{}pounce\sphinxhyphen{}bounce’. ‘Pose’ a question, students talk about it in a pair. After a few minutes, ‘pounce’ on one pair and ask for a response. The ‘bounce’ involves getting other pairs to respond to the response.

\end{itemize}

\sphinxincludegraphics[width=342\sphinxpxdimen,height=228\sphinxpxdimen]{{pgce_geography_assessment_fig04}.jpg}

Figure 4 Hands up or ‘who to choose’?


\subsubsection{3.2 The impact of feedback}
\label{\detokenize{content/session_00/Part_00_03:3.2-The-impact-of-feedback}}
Significant amounts of time and effort are expended on teacher and peer assessment to provide oral and written feedback comments \textendash{} the challenge is to make this feedback effective (specific, encouraging, clear) and to engage students in a dialogue about their work.

Weeden (2005) suggests prompts should encourage immediate improvements. Can you think of some examples?
\begin{itemize}
\item {} 
Reminder prompts \textendash{} \sphinxstyleemphasis{say more about …}

\item {} 
Scaffold prompts \textendash{} \sphinxstyleemphasis{can you explain why} …? (questions)

\item {} 
Example prompts \textendash{} \sphinxstyleemphasis{choose one of these statements or create your own}.

\end{itemize}

You can look for examples of these and other prompts in the next activity, which focuses on the feedback students receive from a teacher.


\paragraph{Activity 6 AfL questioning and groupwork feedback}
\label{\detokenize{content/session_00/Part_00_03:Activity-6-AfL-questioning-and-groupwork-feedback}}
\sphinxstylestrong{Timing: Allow about 30 minutes}


\subparagraph{Question}
\label{\detokenize{content/session_00/Part_00_03:id2}}
Watch the \sphinxhref{https://www.open.ac.uk/libraryservices/resource/video:106748\&f=28688}{‘Group work and feedback’} video. (Alternatively, you can read a transcript.)

Note any issues that are relevant to your own practice or tips that you could develop in your own feedback.

For example:
\begin{itemize}
\item {} 
How do you respond when a student goes beyond the intended learning outcomes?

\item {} 
How do you share feedback? Does it always have to ‘come through you’?

\end{itemize}


\subsubsection{3.3 Self and peer assessment}
\label{\detokenize{content/session_00/Part_00_03:3.3-Self-and-peer-assessment}}
Improving learning through assessment requires students to be involved in their own learning, assessing themselves and understanding what they need to do to improve (Black and Wiliam, 1998). This can be achieved through:
\begin{itemize}
\item {} 
sharing learning goals and assessment criteria with students; involving students in the identification of learning goals

\item {} 
helping students to know and recognise the standards they are aiming for

\item {} 
involving students in self and peer assessment, encouraging them to review their work critically and constructively

\item {} 
providing feedback that helps students recognise their next steps and how to take them

\item {} 
both teacher and students reviewing and reflecting on assessment data.

\end{itemize}

Peer assessment is an important tool in enabling students to develop as independent learners. Independent learners are able to self\sphinxhyphen{}assess effectively and to use metacognitive skills to decide how to move forward, developing self\sphinxhyphen{}regulation and self\sphinxhyphen{}efficacy. Engaging with criteria and giving feedback in peer assessment will help students to assess their own learning. They will also become more engaged in their learning, and will build their confidence in discussing work with peers in a
reflective, collaborative process. To facilitate self and peer assessment you need to commit to learners having control over the process, being able to discuss learning and developing effective student feedback.


\paragraph{Activity 7 Self and peer assessment}
\label{\detokenize{content/session_00/Part_00_03:Activity-7-Self-and-peer-assessment}}
\sphinxstylestrong{Timing: Allow about 40 minutes}


\subparagraph{Question}
\label{\detokenize{content/session_00/Part_00_03:id3}}
Open the geography self\sphinxhyphen{}assessment sheet.

Adapt the sheet for a contrasting topic of your choice.

Note the incorporation of targets in this self\sphinxhyphen{}assessment sheet.
\begin{itemize}
\item {} 
What is the timescale of the targets?

\item {} 
What are the advantages of this approach to targets?

\item {} 
What are the weaknesses?

\end{itemize}


\subparagraph{Question}
\label{\detokenize{content/session_00/Part_00_03:id4}}\label{\detokenize{content/session_00/Part_00_03:id5}}
Watch the video at \sphinxhref{https://www.open.ac.uk/libraryservices/resource/video:106749\&f=28688}{‘Secondary AfL \textendash{} geography’}. (Alternatively, you can read a transcript.)

Observe the variety of support provided to different year groups for peer assessment.

The example includes AfL that is very directly related to AoL, tied to mark schemes and model answers. Can you see any limitations in this approach?


\subparagraph{Question}
\label{\detokenize{content/session_00/Part_00_03:question-2}}\label{\detokenize{content/session_00/Part_00_03:id6}}
Listen to \sphinxhref{https://www.open.ac.uk/libraryservices/resource/video:106752\&f=28688}{Dylan Wiliam} review some of the benefits of self and peer assessment as a key component of effective learning

Student involvement in setting ‘next steps’ and providing feedback aids their learning about geography and about the skills involved in learning. However, be aware that:
\begin{itemize}
\item {} 
next steps identified by students may be limited by their knowledge and understanding

\item {} 
in self or peer assessment unintended outcomes or alternative viewpoints may not be valued

\item {} 
model answers and prescriptive mark schemes may limit student independence for some formats of learning (enquiry or more creative work)

\item {} 
weak literacy can be an issue.

\end{itemize}

You will need to consider issues like these as you support student self and peer assessment.


\subsubsection{3.4 Summary}
\label{\detokenize{content/session_00/Part_00_03:3.4-Summary}}
The type of assessment used influences the type of student learning: a focus on formulaic testing can encourage ‘teaching to the test’ to maximise performance and may result in shallow learning. The intention of AfL is to facilitate geographical learning that is deep and advances student understanding as well as skills and independence (Weeden, 2013, p. 149).


\subsection{4 What are the challenges related to assessment in geography?}
\label{\detokenize{content/session_00/Part_00_04:4-What-are-the-challenges-related-to-assessment-in-geography?}}\label{\detokenize{content/session_00/Part_00_04::doc}}
The preceding sections in this course have already alluded to some challenges related to assessment in geography, including differing definitions of progress and purposes of assessment; an accountability culture that focuses attention on AoL and piecemeal adoption of AfL techniques.

This section will focus on the challenges related to reliability and validity of assessments and to differentiation.


\subsubsection{4.1 Assessment validity and reliability}
\label{\detokenize{content/session_00/Part_00_04:4.1-Assessment-validity-and-reliability}}
When considering whether assessment is fair to all students, two issues are important: validity and reliability. Validity asks whether grades generated by a testing system represent a student’s achievement in the whole of geography. Can a series of timed, written tests at the end of a key stage assess all those things we think are important for students to learn about in school geography? Butt (n.d.) describes AoL as generating ‘high stakes’ results. By contrast, he describes teacher assessment
outputs as ‘low stakes’. Teacher assessment is often seen as being of lower status than the results of tests even though it is more likely to be valid in these terms.

An over\sphinxhyphen{}reliance on test results may lead teachers to make generalisations and judgements about a student’s capability in all aspects of geography, based on the formal testing of a subset. For example, a grade ascribed on a short answer test says nothing about a student’s problem\sphinxhyphen{}solving or creative capacities in geography, nor about their ability to work in groups or engage in extended tasks. Perhaps all that can be said is that the tests simply tell us about the capabilities of students to
answer questions at a particular time and of a particular type (and in the conditions and circumstances of the test) \textendash{} no more and no less.

Reliability asks whether a student’s performance changes (or not) depending on the particular questions that are set. Ideally, assessments should give every student an optimal opportunity to demonstrate what they know. In practice, however, tests have been found to be biased against students from particular backgrounds, socio\sphinxhyphen{}economic classes, ethnic groups or gender (Pullin, 1993; Cooper, 1998). Equity issues are particularly important when assessment results are used to label students or deny
them access to courses or careers in the future.

Research shows that working\sphinxhyphen{}class children:

… go into lower sets …

… are set less challenging test items …

… and therefore they have to answer more ‘realistic’ test items …

… so they achieve lower test results than they deserve …

… so they are confirmed as being appropriately placed in lower sets …

… so they may experience a less rich curriculum, and will learn less content …

… which restricts their opportunities to get high grades in examinations at the end of their schooling …

… which restricts their career opportunities …

… which means they and their family may be locked into a cycle of underachievement.

As you set assessments look for bias that might disadvantage different groups of students. Avoid setting a ceiling in terms of what students can achieve.


\paragraph{Reflection point}
\label{\detokenize{content/session_00/Part_00_04:Reflection-point}}
Are tests used fairly?


\subsubsection{4.2 Differentiation}
\label{\detokenize{content/session_00/Part_00_04:4.2-Differentiation}}
Differentiation is planning to ensure that all students in the class can understand and make progress in their learning. It includes tailoring assessment and feedback to the needs of individual students so that they can make progress and will help to address some of the concerns about reliability and validity of assessment.

Hattie (2012) identifies the following five characteristics of effective differentiation, all of which are relevant to assessment:
\begin{enumerate}
\sphinxsetlistlabels{\arabic}{enumi}{enumii}{}{.}%
\item {} 
All students have the opportunity to explore the key concepts and achieve success.

\item {} 
Frequent formative assessment is used in order to monitor students’ progress.

\item {} 
Teachers are flexible about how they group students, giving them the opportunity to work alone, with different people or as a whole class.

\item {} 
Students are actively engaged in activities that will enable them to achieve success.

\item {} 
Differentiation is related to differential learning gains rather than focusing on attainment levels. Students who are making progress, regardless of their starting point, will need different opportunities from those who are not.

\end{enumerate}

Planning assessment that offers all of these five characteristics is not easy and requires experience and knowledge of what works. However, that should not stop you working to achieve effective differentiation. Using AfL will enable you to find out who is or is not making progress so that you can use assessment to support the learning of all students.


\paragraph{Activity 8 Recognising student diversity \textendash{} inclusive assessment}
\label{\detokenize{content/session_00/Part_00_04:Activity-8-Recognising-student-diversity-_-inclusive-assessment}}
\sphinxstylestrong{Timing: Allow about 40 minutes}


\subparagraph{Question}
\label{\detokenize{content/session_00/Part_00_04:Question}}
How could geographical assessments discriminate against working class students?

You might consider a range of learning and assessment activities, experiences outside school or access to resources.


\subparagraph{Question}
\label{\detokenize{content/session_00/Part_00_04:id1}}
How can you effectively differentiate to avoid such discrimination?


\subparagraph{Question}
\label{\detokenize{content/session_00/Part_00_04:id2}}

\subparagraph{Discussion}
\label{\detokenize{content/session_00/Part_00_04:Discussion}}
There is a wide variety of possible discrimination. Some teachers will adopt a ‘deficit model’ in relation to the experiences, aspirations and abilities of working\sphinxhyphen{}class students. Such assumptions may affect the challenge and the type of assessments used.

At the other end of the scale, some teachers are oblivious to potential barriers to learning related to the socio\sphinxhyphen{}economic situation of students. Limited access to resources and family support may disadvantage some students.

The key to differentiating to avoid discrimination lies in getting to know the students. It is important not to label working\sphinxhyphen{}class students as a homogeneous group.


\subparagraph{Question}
\label{\detokenize{content/session_00/Part_00_04:id3}}
Read each of the short student profiles below. For each student, write notes about how you might recognise their specific needs related to AfL in geography lessons.

Ash has cerebral palsy, which causes difficulty with coordination. He also tires easily.


\subparagraph{Question}
\label{\detokenize{content/session_00/Part_00_04:id4}}
Winston has Asperger’s syndrome. He is fascinated by studying different places but has communication difficulties and finds it difficult to be creative.


\subparagraph{Question}
\label{\detokenize{content/session_00/Part_00_04:id5}}
Nina came to Britain with her parents following a war in her own country. Her parents have since divorced and Nina lives with her mother. She is experiencing emotional difficulties, which impact on her social relationships and her ability to concentrate.


\subparagraph{Question}
\label{\detokenize{content/session_00/Part_00_04:id6}}

\subparagraph{Discussion}
\label{\detokenize{content/session_00/Part_00_04:id7}}
You may have considered:
\begin{itemize}
\item {} 
adapting language, instructions and questions especially for Winston and Nina

\item {} 
completing short assessments throughout a lesson, rather than leaving them until the end (Ash, Nina)

\item {} 
help from a scribe (Ash)

\item {} 
ways of raising the students’ self\sphinxhyphen{}esteem (for example, setting achievable short\sphinxhyphen{}term goals)

\item {} 
supporting the students with appropriate resources (for example, visual resources)

\item {} 
supporting Nina in improving her social relationships (for peer assessment and group work).

\end{itemize}

Assessment of the physical environment would be crucial if you were working outside the classroom with Ash.


\subparagraph{Question}
\label{\detokenize{content/session_00/Part_00_04:id8}}
What other factors might you consider when planning inclusive assessment?

List a range of factors, explain their relevance and suggest effective responses.


\subparagraph{Question}
\label{\detokenize{content/session_00/Part_00_04:id9}}

\subparagraph{Discussion}
\label{\detokenize{content/session_00/Part_00_04:id10}}
A wide range of disabilities and special educational needs of students will affect their assessed work in many ways. Considering the needs of individual students, rather than labelling all students with a condition as being the same, is key to designing and implementing inclusive assessment.

Some students have different preferred learning styles. The format of assessments may affect their performance and progress, so including a variety of assessment formats (perhaps giving students choices) will be more inclusive.

Students with poor recall or poor attendance may be unable to demonstrate higher\sphinxhyphen{}order thinking skills if factual information is not provided. Including resources in assessments can help these students to exhibit skills such as analysis, evaluation and synthesis.


\subsubsection{4.3 Summary}
\label{\detokenize{content/session_00/Part_00_04:4.3-Summary}}
When planning for assessment for learning and assessment of learning, keep issues of validity, reliability and inclusivity in mind. Maintaining a focus on these will help you to integrate assessments that will inform your teaching and help students to make more progress.


\subsection{Conclusion}
\label{\detokenize{content/session_00/Part_00_05:Conclusion}}\label{\detokenize{content/session_00/Part_00_05::doc}}
Through considering the issues raised in this course, you will have realised that some people assume assessment of geography is straightforward and easy. They often have a narrow view of what geography is and may see learning geography as a matter of learning place names and facts. As more people see geography as a way of thinking about the world, identifying patterns and relationships and understanding complex problems, so assessing students’ progress will be seen to be more complex and
multifaceted.

You should now also understand that assessment rarely means using a test or short questions, unless you need to assess the ability to take a test. Assessment should involve getting to the heart of students’ understanding, and hence requires activities that explore and challenge this. AfL involves students as they are the ones who are doing the learning. It should be continuous and integrated into learning activities so that students have the opportunity to act on feedback and you can adapt
teaching and learning activities to more effectively support your students.


\subsection{Keep on learning}
\label{\detokenize{content/session_00/Part_00_06:Keep-on-learning}}\label{\detokenize{content/session_00/Part_00_06::doc}}
\sphinxincludegraphics[width=300\sphinxpxdimen,height=200\sphinxpxdimen]{{ol_skeleton_keeponlearning_image}.jpg}


\bigskip\hrule\bigskip



\subsubsection{Study another free course}
\label{\detokenize{content/session_00/Part_00_06:Study-another-free-course}}
There are more than \sphinxstylestrong{800 courses on OpenLearn} for you to choose from on a range of subjects.

Find out more about all our \sphinxhref{http://www.open.edu/openlearn/free-courses?utm\_source=google\&utm\_campaign=ol\&utm\_medium=ebook}{free courses}.


\bigskip\hrule\bigskip



\bigskip\hrule\bigskip



\subsubsection{Take your studies further}
\label{\detokenize{content/session_00/Part_00_06:Take-your-studies-further}}
Find out more about studying with The Open University by \sphinxhref{http://www.open.ac.uk/courses?utm\_source=google\&utm\_campaign=ou\&utm\_medium=ebook}{visiting our online prospectus}.

If you are new to university study, you may be interested in our \sphinxhref{http://www.open.ac.uk/courses/do-it/access?utm\_source=google\&utm\_campaign=ou\&utm\_medium=ebook}{Access Courses} or \sphinxhref{http://www.open.ac.uk/courses/certificates-he?utm\_source=google\&utm\_campaign=ou\&utm\_medium=ebook}{Certificates}.


\bigskip\hrule\bigskip



\bigskip\hrule\bigskip



\subsubsection{What’s new from OpenLearn?}
\label{\detokenize{content/session_00/Part_00_06:What_u2019s-new-from-OpenLearn?}}
\sphinxhref{http://www.open.edu/openlearn/about-openlearn/subscribe-the-openlearn-newsletter?utm\_source=google\&utm\_campaign=ol\&utm\_medium=ebook}{Sign up to our newsletter} or view a sample.


\bigskip\hrule\bigskip


For reference, full URLs to pages listed above:

OpenLearn \textendash{} \sphinxhref{http://www.open.edu/openlearn/free-courses?utm\_source=google\&utm\_campaign=ol\&utm\_medium=ebook}{www.open.edu/openlearn/free\sphinxhyphen{}courses}

Visiting our online prospectus \textendash{} \sphinxhref{http://www.open.ac.uk/courses?utm\_source=google\&utm\_campaign=ou\&utm\_medium=ebook}{www.open.ac.uk/courses}

Access Courses \textendash{} \sphinxhref{http://www.open.ac.uk/courses/do-it/access?utm\_source=google\&utm\_campaign=ou\&utm\_medium=ebook}{www.open.ac.uk/courses/do\sphinxhyphen{}it/access}

Certificates \textendash{} \sphinxhref{http://www.open.ac.uk/courses/certificates-he?utm\_source=google\&utm\_campaign=ou\&utm\_medium=ebook}{www.open.ac.uk/courses/certificates\sphinxhyphen{}he}

Newsletter ­\textendash{} \sphinxhref{http://www.open.edu/openlearn/about-openlearn/subscribe-the-openlearn-newsletter?utm\_source=google\&utm\_campaign=ol\&utm\_medium=ebook}{www.open.edu/openlearn/about\sphinxhyphen{}openlearn/subscribe\sphinxhyphen{}the\sphinxhyphen{}openlearn\sphinxhyphen{}newsletter}



\renewcommand{\indexname}{Index}
\printindex
\end{document}